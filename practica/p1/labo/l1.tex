\documentclass[12pt]{article}
\usepackage[margin=1.0in]{geometry}

\usepackage{g-util}
\usepackage{amsfonts, amsmath, amssymb}

\title{Organizaciòn del Computador II
\large Labo 1}
\author{F. Galileo Cappella Lewi}
\date{1c2022}

\begin{document}

\begin{enumerate}
\item \begin{enumerate} %EJ: 1

\item 32-bits
\item \(2^32\) bytes = 4GB
\item 8, 32-bits
\item 32-bits, marca la posición de la siguiente instrucción
\item Porqe tiene qe poder "llevarte" a cualquier lugar en la memoria

\end{enumerate}

\item \begin{enumerate} %EJ: 2

\item 32-bits, y guarda los status flags (resultado de operaciones aritméticas), un flag de control ("endianess" de los strings), y flags de sistema (activan y desactivan cosas del OS)
\item \begin{itemize} \
	\item[Overflow:] 11, Prendido si el número es de un valor muy grande para el destino 
	\item[Sign:] 7, Vale lo mismo qe el bit más significativo del resultado
	\item[Interrupt:] 9, Marca si el procesador va a responder a interrupciones
\end{itemize}
\item No es lo mismo, los primeros 32bits están reservados. Y cambia de nombre

\end{enumerate}

\item \begin{enumerate} %EJ: 3

\item Se usa para pasar parámetros entre subrutinas, y para guardar valores dentro de mi subrutina
\item El procesador lo ubica en la memoria (en base al necesitado). Y dentro de mi memoria asignada, arriba de todo
\item Apunta al último elemento +1 de mi stack frame 
\item Apunta a una posición dentro del stack, se suele usar para marcar la base del stack frame

ebp: 6 \\
esp: 4 \\

13: \\
12: \\
11: \\
10:+EIP \\
9:+EBP <- MI EBP \\
8:+ <- nEBP, nESP \\
7: blabla \\
6:-EIP <- CALL \\
5:-EBP <- yo, de buena onda \\
4:-llamada <- EBP, ESP \\
3:- \\
2:- \\
1: blabla \\
0: blabla \\

\item Se guarda el valor del EIP para poder continuar una vez terminada la subrutina. Y también se suele guardar el EBP
\item Pop-ea el ultimo valor del stack (asumiendo qe es el EIP guardado) y lo guarda en el registro del EIP
\item Hay qe asegurarse qe sea el EIP guarado, se suele guardar en la posición del EBP (qe a su vez suele ser la base del stack)
\item Puede tener un ancho de 16-bits o de 32-bits. En 64-bits tiene de 16-bits o 64-bits 
\item \begin{itemize} \ \\

\item[Galileo] Sí, pero habría qe asegurarse otra forma de ubicar el EIP y una forma de saber cuánto del stack usé. CONVENCIDO: Si está resuelto por afuera mio que no se me llene el stack, no hace falta saber cuánto stack usé
\item[Luciana] Acuerda con Galileo
\item[Juan] Sí, pero depende del uso qe necesitás. (eg. no necesitas voler al EIP)

\end{itemize}

\end{enumerate}

\item \begin{enumerate} %EJ4

\item 

\begin{table}[]
\begin{tabular}{l|lll}
  & a & b & c \\
 DEC & 1 operando, r/m & decrementa en 1 su valor &  \\
 ADD & 2 operandos, r/m e imm/r & le suma el segundo operando al valor en la posición del primero & 		 \\
 MOV & 2 operandos, r/m y r/m & copia el valor del segundo operando al primer operando &  \\
 JZ & 1 operando & Si el flag 0 está marcado (aka. el resultado de la ultima operación aritmética fue 0) salta a la posición & %TODO: Tipo del operando, y a donde salta
\end{tabular}
\end{table}

; Calcula 9 * 3 \\
MOV EBX, 9 ; Valor \\
MOV ECX, 3 ; *3, lo uso de contador para las sumas \\
MOV EAX, 0 ; Empiezo en cero \\
loop: \\
	ADD EAX, EBX \\
	DEC ECX \\
	JZ end \\
	JMP loop \\
end: \\
	MOV 0x0000000000000022, EAX ; Resultado en la posicion 42 de la memoria \\

\end{enumerate}

\end{enumerate}
\end{document}
