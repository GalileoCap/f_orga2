\begin{enumerate}
\item El modo real existe por legacy, permite operar con las limitaciones del 8086 (16-bits, 1MB de memoria, es inseguro, acceso a todas las instrucciones, etc.). La computadora arranca en este modo. \\
  Mientras que el modo protegido es el usado normalmente, tiene hasta 4GB de memoria disponible, 4 niveles de protección, y niveles de privilegios.
\item Pasamos a modo protegido porque: Permite hacer programas más grandes, y manejar los privilegios de los programas (p.e. evitar que cualquiera lean cualquier pedazo de memoria). \\
  Sí podríamos, de hecho los sistemas operativos del 8086 trabajaban en modo rea.
\item La GDT es una tabla que lleva registro de los segmentos definidos. Los segment descriptors estan formados de diferentes partes: \begin{itemize}
  \item Base address: es la posición en memoria del byte 0 de este segmento.
  \item Segment limit: es la última posición indexable dentro del segmento.
  \item Type: Marca para qé va a ser usado el segmento %TODO
  \item S: 1 si va a ser usado para código o datos, 0: si va a ser usado por el sistema
  \item DPL: Marca el nivel de privilegio necesario para acceder a este segmento
  \item P: Marca si el segmento está presente o no en la computadora, si se indexa a un segmento no-presente salta una excepción (discos externos)
  \item D/B: 0 el segmento es de 16-bits, 1 el segmento es de 32-bits %TODO: No 64?
  \item G: 0: TrueLimit = (Limit + 1), 1: TrueLimit = (Limit + 1)*4KB
  \item L: En modo IA-32e se usa para marcar si las instrucciones en este segmento usan 64-bits o 32-bits.
  \item El bit 20 está libre para el sistema.
  \end{itemize}
\item 0b1010 %TODO conforming, accessed
\item \begin{itemize}
  \item[a:] Base0: 0000 0000, G: 1, D/B: , L: , Limit0:, P:, DPL:, S:, Type:1010, Base1:0000 0000, Base2:0000 0000 0000 0000, Limit1:\\
    Final: 0b0000 0000
  \end{itemize}
\end{enumerate}
