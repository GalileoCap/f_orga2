\begin{document}

\begin{enumerate}
\item La convención de llamada de C estipula cómo se hacen los llamados a funciones, para solucionar problemas como mantener los datos del contexto anterior a la llamada, pasar parámetros, y devolver resultados.
  \paragraph{64-bits} \begin{itemize}
    \item Hay 5 registros no volátiles (tengo qe asegurarme qe sigan valiendo lo mismo al final de mi llamada).
    \item Si tengo como retorno a un entero o a un puntero lo guardo en RAX, si es un flotante en XMMO.
    \item Al final de mi llamada tengo qe asegurarme qe el stack esté como cuando me llamaron.
    \item Antes de llamarme el stack tiene qe estar alineado a 8-Bytes (16-Bytes si uso libc).
    \item Los parámetros se pasan en 6 registros para enteros, y 8 registros para floats, y si hay más se pasan de derecha a izquierda a través del stack.
  \end{itemize}
  \paragraph{32-bits} \begin{itemize}
    \item Hay 4 registros no volátiles.
    \item El valor de retorno se guarda en EAX.
    \item Al final de mi llamada tengo qe asegurarme qe el stack esté como cuando me llamaron.
    \item Antes de llamarme el stack tiene qe estar alineado a 4-Bytes.
    \item Los parámetros se pasan de derecha a izquierda a través del stack.
  \end{itemize}

\item En C se encarga el compilador, y en ASM le programadore.

\item \paragraph{Stack Frame} Todo lo pertinente a la función llamada (parámetros, rip, rbp, etc.)
  \paragraph{Prólogo} Se prepara todo para la llamada (guardar espacio en el stack, alinear a 8/16-Bytes, guardar registros no-volátiles)
  \paragraph{Epílogo} (restaurás registros no-volátiles, reseteamos la pila a como estaba antes de llamarnos, guardamos el resultado en el registro correcto, etc.)

\item En el prólogo me guardo un pedazo de memoria, y guardo mis variables temporales ahí

\item 16-bytes, 8-bytes (16-bytes + el tamaño del RIP)

\item Se puede romper :c la interacción, porqe puede qe guarde el resultado en registros diferentes a los esperados, o qe espere qe guarde más parámetros en el stack...
\end{enumerate}

\end{document}
